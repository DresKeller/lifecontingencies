
\documentclass[a4paper]{article}
\usepackage[OT1]{fontenc}
\usepackage{Sweave}
\usepackage{myVignette}
%\bibliographystyle{plainnat}
%\VignetteIndexEntry{An introduction to lifecontingencies package}
%\VignetteKeywords{vig1}
%\VignettePackage{lifecontingencies}
\begin{document}

\title{An introduction to lifecontingencies package}
\author{Giorgio A. Spedicato}

\maketitle

\section{Overview}
I've decided to submit to the cran package to perform life contingencies calculation in order to fill a current
lack within the CRAN archive. I will fill this  vignette further, nevertheless I anticipate the structure:

\begin{enumerate}
	\item Section \ref{sec:package} provides an overview of R usage within actuarial 
	fields and describes the package structure.
	\item Section \ref{sec:examples} gives a wide choice of lifecontingencies packages example.
	\item Finally section \label{sec:discussion} will provide a discussion of results and further potential
	developments.

\end{enumerate}

\section{Lifecontingencies package description} \label{sec:package}

\subsection{Current R actuarial packages}


\subsection{The structure of the package}

\clearpage
\newpage


\section{Examples} \label{sec:examples}

\subsection{Classical financial mathematics example}

\subsection{Working with lifetable objects}

\subsection{Classical actuarial mathematics examples}

\subsection{A thorough examples}

\clearpage
\newpage

\section{Discussion} \label{sec:discussion}

\subsection{Analysis of software capabilities}
\subsection{Prospective developments}

%\bibliography{packagebiblio}

\end{document}
